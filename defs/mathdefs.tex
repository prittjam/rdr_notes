\DeclareMathOperator{\rank}{rank}%
\DeclareMathOperator{\vect}{vec}%
\DeclareMathOperator{\atantwo}{atan2}%
\DeclareMathOperator*{\argmin}{argmin}%
\DeclareMathOperator*{\argmax}{argmax}%

\DeclarePairedDelimiter{\abs}{\lvert}{\rvert}
\DeclarePairedDelimiter{\norm}{\lVert}{\rVert}
\DeclarePairedDelimiter{\innerproduct}{\langle}{\rangle}

\DeclarePairedDelimiter{\parenfences}{(}{)}
\DeclarePairedDelimiter{\brackets}{[}{]}
\DeclarePairedDelimiter{\coord}{(}{)}
\DeclarePairedDelimiterX{\sksym}[1]{[}{]_{\times}}{#1}

% just to make sure it exists
\providecommand\given{}
% can be useful to refer to this outside \Set
\newcommand\SetSymbol[1][]{%
\nonscript\:#1\vert
\allowbreak
\nonscript\:
\mathopen{}}

\DeclarePairedDelimiterX\set[1]\{\}{%
\,\renewcommand\given{\SetSymbol[\delimsize]}
#1\,
}

\NewDocumentCommand\Zz{}{\mathbb{Z}} % Integsers
%\NewDocumentCommand\Re{}{\mathbb{R}} % Reals
\NewDocumentCommand\Rp{}{\mathbb{RP}} % Real-Projective Space

\NewDocumentCommand\Lgn{}{\mathcal{L}} % Lagrangian

\NewDocumentCommand{\vertbar}{}{\rule[-1ex]{0.5pt}{2.5ex}}
\NewDocumentCommand{\horzbar}{}{\rule[.5ex]{2.5ex}{0.5pt}}

\NewDocumentCommand\sqnorm{m}{\norm{#1}^2} % inverse transpose
\NewDocumentCommand\sqltwonorm{m}{\norm{#1}_2^2} % inverse transpose
\NewDocumentCommand\inv{}{\mathsf{-1}} % inverse transpose
\NewDocumentCommand\T{}{\top} % transpose
\NewDocumentCommand\invT{}{-\top} % inverse transpose
\NewDocumentCommand\diag{}{\operatorname{diag}\parenfences}

\NewDocumentCommand\dist{O{}}{d_{\text{#1}}\parenfences}
\NewDocumentCommand\sqdist{O{}}{d_{\text{#1}}^2\parenfences}
\NewDocumentCommand\gaussian{}{\mathcal{N}\parenfences}

\DeclareMathOperator{\EE}{\mathbb{E}}

\newcommand*\rot{\rotatebox{90}}

\NewDocumentCommand\cspond{mm}{#1\leftrightarrow#2}

\NewDocumentCommand\removebslash{m}{%
  {\catcode92=9 \endlinechar-1 \scantokens{#1}}%
}

\NewDocumentCommand\vecbold{m}{\bm{#1}}

\ExplSyntaxOn

\NewDocumentCommand\definevecbold{m}
{
  \clist_map_inline:nn { #1 }
  {
    \cs_new_protected:cpn { vb##1 } { \vecbold{##1} }
  }
}

\NewDocumentCommand{\defineset}{m}
{
  \clist_map_inline:nn { #1 }
  {
    \cs_new_protected:cpn { s##1 } { \mathcal{##1} }
  }
}

\clist_new:N \vectr_clist
\NewDocumentCommand{\vectr}{O{\\}mO{b}}{
  \clist_set:Nn \vectr_clist {#2} % Set the list
  \begin{#3matrix}
  \clist_use:Nn \vectr_clist {#1} % show it with separator from #1 (\\)
  \end{#3matrix}
}

\NewDocumentCommand{\rvec}{mO{b}}{\vectr[&]{#1}[#2]}
\NewDocumentCommand{\cvec}{mO{b}}{\vectr{#1}[#2]}

\NewDocumentCommand{\rmat}{mO{b}}{
  \clist_set:Nn \l_rmat_clist {#1} % Set the list
  \begin{#2matrix}
    \horzbar &
    \clist_use:Nn \l_rmat_clist {& \horzbar \\ \horzbar &}  % show it with separator from #1 (\\)
    & \horzbar
  \end{#2matrix}
}

\NewDocumentCommand{\cmat}{mO{b}}{
  \clist_set:Nn \l_cmat_clist {#1} % Set the list
  \begin{#2matrix}
    & \clist_map_inline:Nn \l_cmat_clist { \vertbar & } \\
    & \clist_use:Nn \l_cmat_clist {& } \\ % Set the list
    & \clist_map_inline:Nn \l_cmat_clist { \vertbar & } \\
  \end{#2matrix}
}

\ExplSyntaxOff

\definevecbold{A,B,C,D,E,F,H,I,J,K,M,P,Q,R,T,U,V,W,X,Y,Z,a,b,c,d,e,f,h,l,m,n,o,p,r,t,u,v,x,y,z}
\defineset{A,B,C,D,E,F,G,H,I,J,K,L,M,N,O,P,Q,R,S,T,U,V,W,X,Y,Z}

\NewDocumentCommand\linf{}{\vbl_\infty}
\NewDocumentCommand\vbSigma{}{\vecbold{\Sigma}}
\NewDocumentCommand\vbphi{}{\vecbold{\phi}}
\NewDocumentCommand\vbzero{}{\vecbold{0}}
\NewDocumentCommand\vbdelta{}{\vecbold{\delta}}
\NewDocumentCommand\vbepsilon{}{\vecbold{\epsilon}}
\NewDocumentCommand\vbmu{}{\vecbold{\mu}}

\def\mKRt{\ensuremath{\mK \begin{bmatrix} \mR & \vt \end{bmatrix}}}

\newcommand*\Let[2]{\State #1 $\gets$ #2}

\def\mupinvof#1{{{#1}^{-\lambda}}}

\newcommand{\eps}{\varepsilon}
\providecommand{\homogvec}[1]{\binom{n}{1}}
\providecommand{\dotprod}[2]{\left\langle #1, #2 \right\rangle}

\newenvironment{smallpmatrix}{\left(\begin{smallmatrix}}{\end{smallmatrix}\right)}


\newcommand{\cmark}{\ding{51}}%
\newcommand{\xmark}{\ding{55}}%
\setlength\aboverulesep{0pt}
\setlength\belowrulesep{0pt}

\newtheorem*{remark}{Remark}
\newtheorem*{desiderata}{Desiderata}
\newtheorem*{result}{Result}


